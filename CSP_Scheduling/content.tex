% DO NOT COMPILE THIS FILE DIRECTLY!
% This is included by the other .tex files.

\begin{frame}[t,plain]
\titlepage
\end{frame}

\begin{frame}
  \frametitle{Outline}
  \tableofcontents
\end{frame}

\AtBeginSubsection[]                            % 在每个子段落之前
{
  \begin{frame}
    \frametitle{Table of Contents}
    \setlength{\parskip}{0\baselineskip}
    \tableofcontents[current,currentsubsection]
    \setlength{\parskip}{0.5\baselineskip}
  \end{frame}
}

\AtBeginSection[]                            % 在每个子段落之前
{
  \begin{frame}
    \frametitle{Table of Contents}
    \setlength{\parskip}{0\baselineskip}
    \tableofcontents[current,currentsubsection]
    \setlength{\parskip}{0.5\baselineskip}
  \end{frame}
}

\section{Scheduling Problem}
\begin{frame}[t]{Problem Background}
  \begin{itemize}[<+->]
    \item Determine a work schedule for assistants that is both reasonable
       and efficient.
    \item Entities:
      \begin{itemize}
        \item Assistants set $A$
        \item Time period set $T$
      \end{itemize}
    \item Requirements:
      \begin{itemize}
        \item efficient: There should be sufficient assistants in any specific
          time periods.
        \item reasonable:
          \begin{itemize}
            \item Assistants should not work when they are having courses.
            \item Assistants should not work at multiple places in the meantime.
            \item Work times among students should not vary widely.
          \end{itemize}
      \end{itemize}
  \end{itemize}
\end{frame}

\begin{frame}[t]{Modeling}
  How can we describe the problem in mathematical model?
  \begin{itemize}[<+->]
    \item Let's review the problem again.
      \begin{itemize}
        \item VARIABLES: assistant set $A$ consisting of $n$ assistants, time
          period set $T$ consisting of $m$ time period, the i-th assistant $A_i$
          has some free time period $F_i$, where $F_i \in T$
        \item OPERATION: select a certain free time $F$, and discard the
          others
        \item TARGET: Work time for an assistant and quantity of assistant of a
          time period are in a specific range. No assistant will simultaneously
          work in overlapping time period.
      \end{itemize}
    \item Method: construct a matrix, with assistants as rows, time periods as
      column.

  \end{itemize}
\end{frame}

\begin{frame}[t]{Matrix Notation}
$$\begin{pmatrix}
  1& 1& 1& 0\\
  1& 0& 1& 1\\
  1& 1& 1& 1\\
  1& 0& 1& 1
  \end{pmatrix}$$
  \begin{itemize}
    \item A matrix $M$ representing the free time period.
    \item OPERATION: change a certain entries from 1 to 0.
    \item TARGET: $\forall j \in [1, m], \sum_{i=1}^nM_{ij}$ and $\forall i \in
      [1, n], \sum_{j=1}^mM_{ij}$
  \end{itemize}
\end{frame}

\section{Algorithm}
\begin{frame}[t]{Brainstorm}
  \begin{center}
  \huge{What will you probably do to solve this problem?}
  \end{center}
\end{frame}

\begin{frame}[t]{Evaluation}
  How to evaluate an algorithm?
  \begin{itemize}[<+->]
    \item Efficiency?
    \item Flexibility
    \item Extendability
  \end{itemize}
\end{frame}

\begin{frame}[t]{First Attempt: Genetic Algorithm}
  \begin{itemize}
    \item An algorithm adapting Darwin's evolution theory by natural selection
    \item Steps
      \begin{itemize}
        \item Initialization
        \item Evaluation
        \item Selection
        \item Crossover
        \item Mutation
        \item Termination
      \end{itemize}
    \item Pros \& Cons
  \end{itemize}
\end{frame}

\begin{frame}[t]{Second Attempt: Encoding into Constraint Satisfaction Problem}
  \begin{itemize}[<+->]
    \item A constraint satisfaction problem is defined as a triple $\langle X,D,C
      \rangle$, where
      \begin{itemize}
          \item $X = \{X_1, \ldots,X_n\}$ is a set of variables,
          \item $D = \{D_1, \ldots, D_n\}$ is a set of the respect domains of values,
          and
          \item $C = \{C_1, \ldots, C_m\}$ is a set of constraints.
      \end{itemize}
    \item Now we can encode sceduling problem into CSP.
    \item Typically, There are two type of algorithms.
      \begin{itemize}
        \item complete search
        \item incomplete search, or local search
      \end{itemize}
  \end{itemize}
\end{frame}

\begin{frame}[t]{Local Search}
  A general framework of local search:
  \begin{itemize}
    \item Randomly generate a solution.
    \item find a possible better variable to flip.
    \item if stuck in a local optima, restart the procedure.
  \end{itemize}

  Local search has cycling problem!
\end{frame}

\section{Extensions}
\begin{frame}[t]{Extensions}
  \begin{itemize}[<+->]
    \item Nurse Scheduling Problem
    \item Course Scheduling Problem
  \end{itemize}
\end{frame}
