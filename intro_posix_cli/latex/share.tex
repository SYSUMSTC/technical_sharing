\documentclass[aspectratio=169]{beamer}
\usetheme[left,hideothersubsections]{AAUsidebar}
\usepackage{lmodern}
\usepackage{amssymb,amsmath}
\usepackage{ifxetex,ifluatex}
\usepackage{fixltx2e} % provides \textsubscript
\usepackage[T1]{fontenc}
\usepackage[utf8]{inputenc}
\usepackage[english]{babel}
\usepackage{helvet}
\usepackage{upquote}
\usepackage{microtype}
\UseMicrotypeSet[protrusion]{basicmath} % disable protrusion for tt fonts

%\definecolor{aau}{RGB}{33,26,82}
\definecolor{aau}{RGB}{35,50,150}

\setlength{\emergencystretch}{3em}  % prevent overfull lines

\title{POSIX Command Line Utilities}
\subtitle{Just A Glimpse}
\author[Siyuan Liu]{Siyuan Liu\\\href{mailto:leasunhy@gmail.com}{{\tt leasunhy@gmail.com}}}
\date{Nov 22\textsuperscript{nd}, 2015}

\institute[
    Microsoft Student Technology Club\\
    Sun Yat-sen University
]{
    Microsoft Student Technology Club\\
    Sun Yat-sen University
}

\setbeamercolor{sidebar}{fg=blue!20}
\setbeamercolor{frametitle}{bg=aau}

\begin{document}
{\aauwavesbg
\begin{frame}[plain,noframenumbering]
    \setcounter{tocdepth}{1}
    \titlepage
\end{frame}
}

% TOC
\begin{frame}{Agenda}{}
\tableofcontents
\end{frame}

%%%%%%%%%%%%%%%%%%%%%%%%%%%%%

\section{Prologue}\label{prologue}

%%%%%%%%%%%%%%%%%%%%%%%%%%%%%
\subsection{Terminology}
\begin{frame}{Prologue}{Some Terminology}

\begin{itemize}[<+->]
\item
  {\bf POSIX}: \textbf{P}ortable \textbf{O}perating \textbf{S}ystem \textbf{I}nterface
\item
  {\bf CLI}: \textbf{C}ommand \textbf{L}ine \textbf{I}nterface
\item
  Part of \textbf{POSIX} are CLI utilities, the subject today.
\item
  {\bf kernel}: The kernel of OS is a program that manages resources and implement IPC.
\item
  {\bf shell}: Opposite of kernel. The machine/human interface.
\end{itemize}

\end{frame}
%%%%%%%%%%%%%%%%%%%%%%%%%%%%%

%%%%%%%%%%%%%%%%%%%%%%%%%%%%%
\subsection{Motivation}
\begin{frame}{Prologue}{Motivation}

\begin{itemize}[<+->]
\item
  Solve various tasks efficiently.
\item
  Solve programming-needed tasks, without programming.
\item
  Functional programming like experience: simple utilities pipelined to
  solve complex problems.
\item
  Most servers don't have a GUI, so CLI may be your only choice.
\item
  \textbf{Bigger} than \textbf{bigger}.
\item
  ... Just bacause it's a lot of fun :-)
\end{itemize}

\end{frame}
%%%%%%%%%%%%%%%%%%%%%%%%%%%%%

\section{Shell Basics}\label{shell-basics}

%%%%%%%%%%%%%%%%%%%%%%%%%%%%%%%
\subsection{Basics of Basics}
\begin{frame}{Shell Basics}{Basics of Basics}

\begin{itemize}[<+->]
\item
  \texttt{cd}, \texttt{pwd}, \texttt{ls}
\item
  \texttt{cp}, \texttt{mv}, \texttt{rm}
\item
  \texttt{echo}
\item
  Running an executable.
\item
  Interruptting a running command: \texttt{Ctrl+C}.
\item
  Input an \texttt{End Of File}: \texttt{Ctrl+D}.
\item
  Suspending a running command: \texttt{Ctrl+Z}.
\item
  ... and resuming: \texttt{fg}.
\end{itemize}

\end{frame}
%%%%%%%%%%%%%%%%%%%%%%%%%%%%%

%%%%%%%%%%%%%%%%%%%%%%%%%%%%%%%
\subsection{Arguments}
\begin{frame}{Shell Basics}{Arguments}

\begin{itemize}[<+->]
\item
  \texttt{ls\ /etc} lists the contents of \texttt{/etc}
\item
  \ldots{} where \texttt{ls} is the name of the command,
\item
  \ldots{} and \texttt{/etc} is the \textbf{SECOND} argument
  (argv{[}1{]}).
\item
  Where has the \textbf{FIRST} argument gone?
\item
  It's the name of the executable:\\
  here, the first argument is \texttt{ls}.
\end{itemize}

\end{frame}
%%%%%%%%%%%%%%%%%%%%%%%%%%%%%

%%%%%%%%%%%%%%%%%%%%%%%%%%%%%%%
\subsection{Key Bindings}
\begin{frame}{Shell Basics}{Key Bindings (Emacs-like)}

\begin{itemize}[<+->]
\item
  Completion: \texttt{Tab}
\item
  Fast cursor movement:\\
  \texttt{Ctrl/Alt+F}, \texttt{Ctrl/Alt+B}, \texttt{Ctrl+A/E}
\item
  Repeat: \texttt{Alt+A}.
\item
  Clear: \texttt{Ctrl+L}
\item
  Fast line editing:\\
  \texttt{Alt+D\ /\ Ctrl+W}, \texttt{Alt+W\ /\ Ctrl+K},
  \texttt{Ctrl+H/D}
\item
  Reverse search: \texttt{Ctrl+R}\\
  -- it is awesome!
\end{itemize}

\end{frame}
%%%%%%%%%%%%%%%%%%%%%%%%%%%%%

%%%%%%%%%%%%%%%%%%%%%%%%%%%%%%%
\subsection{Glob}
\begin{frame}{Shell Basics}{Glob}

\begin{itemize}[<+->]
\item
  \texttt{*}: matches anything, greedily.
\item
  \texttt{\{\}}:\\
  \texttt{abc\{1,\ 2\}} expands into \texttt{abc1\ abc2}\\
  \texttt{abc\{1..10\}} expands into
  \texttt{abc1\ abc2\ abc3\ abc4\ ...\ abc10}
\end{itemize}

\end{frame}
%%%%%%%%%%%%%%%%%%%%%%%%%%%%%

%%%%%%%%%%%%%%%%%%%%%%%%%%%%%%%
\subsection{Pipes}
\begin{frame}{Shell Basics}{Pipes}

\begin{itemize}
\item
  Example: \texttt{echo\ 1\ 2\ \textbar{}\ ./a.out}
\item
  Example: \texttt{./a.out} and then input \texttt{1\ 2}
\end{itemize}

\end{frame}
%%%%%%%%%%%%%%%%%%%%%%%%%%%%%

%%%%%%%%%%%%%%%%%%%%%%%%%%%%%%%
\subsection{IO Redirection}
\begin{frame}{Shell Basics}{Input/output Redirection}

\begin{itemize}
\item
  \texttt{\textgreater{}}, \texttt{\textgreater{}\textgreater{}}
\item
  \texttt{\textless{}}
\item
  Special: \texttt{\textless{}\textless{}}.
\end{itemize}

\end{frame}
%%%%%%%%%%%%%%%%%%%%%%%%%%%%%

%%%%%%%%%%%%%%%%%%%%%%%%%%%%%%%
\subsection{Special Files}
\begin{frame}{Shell Basics}{Special files}

\begin{itemize}[<+->]
\item
  \texttt{/dev/stdin}: stdin. Any explanation needed?
\item
  \texttt{/dev/stdout}: stdout. ?
\item
  \texttt{/dev/zero}, \texttt{/dev/one}: zero and one.
\item
  \texttt{/proc/meminfo}, \texttt{/proc/cpuinfo}:\\
  Memory/CPU information.
\item
  \texttt{/proc/\textless{}id:int\textgreater{}}:\\
  Information of the process with pid \texttt{id}.
\item
  \texttt{/dev/null}:\\
  BLACK HOLE.
\end{itemize}

\end{frame}
%%%%%%%%%%%%%%%%%%%%%%%%%%%%%

%%%%%%%%%%%%%%%%%%%%%%%%%%%%%%%
\subsection{\tt /dev/null}
\begin{frame}{Shell Basics}{\texttt{/dev/null}: Black Hole Explained}

\begin{itemize}[<+->]
\item
  What black hole?
\item
  Why black hole?
\item
  Example: \texttt{cat\ \textgreater{}\ /dev/null}
\item
  Jokes:
\item
  Please send complains to /dev/null.
\item
  I've learned all my courses and remembered them using my /dev/null.
\item
  I surely saved my money from 11.11. I put them in my /dev/null.
\end{itemize}

\end{frame}
%%%%%%%%%%%%%%%%%%%%%%%%%%%%%

%%%%%%%%%%%%%%%%%%%%%%%%%%%%%%%
\subsection{General Usage}
\begin{frame}{Shell Basics}{General usage}

\begin{itemize}[<+->]
\item
  \texttt{man}: Read manual for the command.
\item
  \texttt{apropos}: Search manual page title and description.
\item
  \texttt{awesome-command\ -\/-help}: Show help for
  \texttt{awesome-command}.
\end{itemize}

\end{frame}
%%%%%%%%%%%%%%%%%%%%%%%%%%%%%

\section{Basic Utilities}\label{basic-utilities}

%%%%%%%%%%%%%%%%%%%%%%%%%%%%%%%
\subsection{\tt cat}
\begin{frame}{Basic Utilities}{\texttt{cat} - concatenate file(s)}

\begin{itemize}[<+->]
\item
  Example: \texttt{cat\ 1.txt}
\item
  Example: \texttt{cat\ 1.txt\ 2.txt}
\item
  Use case: stay tuned.
\end{itemize}

\end{frame}
%%%%%%%%%%%%%%%%%%%%%%%%%%%%%

%%%%%%%%%%%%%%%%%%%%%%%%%%%%%%%
\subsection{\tt paste}
\begin{frame}{Basic Utilities}{\texttt{paste} - join lines of files}

\begin{itemize}[<+->]
\item
  Example: \texttt{paste\ 1.txt\ 2.txt}
\item
  Use case: \texttt{paste\ accounts\ passwords}
\end{itemize}

\end{frame}
%%%%%%%%%%%%%%%%%%%%%%%%%%%%%

%%%%%%%%%%%%%%%%%%%%%%%%%%%%%%%
\subsection{\tt head/tail}
\begin{frame}{Basic Utilites}{\texttt{head} \& \texttt{tail} - first/last n lines of
file}

\begin{itemize}[<+->]
\item
  Example: \texttt{head\ longlongfile.log}
\item
  Example: \texttt{head\ -n\ 20\ longlongfile.log}
\item
  Example: \texttt{tail\ -f\ takeslongtime.log}
\end{itemize}

\end{frame}
%%%%%%%%%%%%%%%%%%%%%%%%%%%%%

%%%%%%%%%%%%%%%%%%%%%%%%%%%%%%%
\subsection{\tt sort}
\begin{frame}{Basic Utilites}{\texttt{sort} - sort lines of file}

\begin{itemize}[<+->]
\item
  Example: \texttt{sort\ accounts}
\item
  Example: \texttt{sort\ -n\ nums}
\item
  Practice:\\
   \texttt{du\ -d\ 1} can be used to find disk usage of every
  subdirectory in current working directory.\\
   So, how can I find out the most space-consuming folder?
\item
  Answer:
  \texttt{du\ -d\ 1\ \textbar{}\ sort\ -n\ \textbar{}\ tail\ -n\ 2}
\item
  How can you achieve the same using GUI, for instance, on Windows?
\item
  Maybe \textbf{grouping}? \(O(1)\) vs \(O(logn)\) :-)
\end{itemize}

\end{frame}
%%%%%%%%%%%%%%%%%%%%%%%%%%%%%

%%%%%%%%%%%%%%%%%%%%%%%%%%%%%%%
\subsection{\tt uniq}
\begin{frame}{Basic Utilites}{\texttt{uniq} - remove duplicated lines}

\begin{itemize}[<+->]
\item
  Example: \texttt{uniq\ dup.txt}
\end{itemize}

\end{frame}
%%%%%%%%%%%%%%%%%%%%%%%%%%%%%

%%%%%%%%%%%%%%%%%%%%%%%%%%%%%%%
\subsection{\tt wc}
\begin{frame}{Basic Utilites}{\texttt{wc} - count words/lines of files}

\begin{itemize}[<+->]
\item
  Example: \texttt{wc\ -l\ cool.cpp}
\end{itemize}

\end{frame}
%%%%%%%%%%%%%%%%%%%%%%%%%%%%%

%\section{Practise Time}\label{practise-time--}

%%%%%%%%%%%%%%%%%%%%%%%%%%%%%%%
\subsection{Practice: Problem 1}
\begin{frame}{Practise Time :-)}{Problem 1}

\begin{itemize}[<+->]
\item
  Problem Statement:\\
  I'm in a folder, which contains numerous C++ source files.\\
  I want to know how many lines of C++ code reside in this folder.
\item
  Answer: \texttt{wc\ -l\ *.cpp}
\end{itemize}

\end{frame}
%%%%%%%%%%%%%%%%%%%%%%%%%%%%%

%%%%%%%%%%%%%%%%%%%%%%%%%%%%%%%
\subsection{Practice: Problem 2}
\begin{frame}{Practise Time :-)}{Problem 2}

\begin{itemize}[<+->]
\item
  Problem Statement:\\
  For some reason (explained later), I want to concatenate all the c++
  files in the current folder into a single file.
\item
  Answer: \texttt{cat\ *.cpp\ \textgreater{}\ total.cpp}
\end{itemize}

\end{frame}
%%%%%%%%%%%%%%%%%%%%%%%%%%%%%

\section{Intermidiate Utilities}\label{Intermidiate-tools}

%%%%%%%%%%%%%%%%%%%%%%%%%%%%%%%
\subsection{\tt xargs}
\begin{frame}{Intermidiate Utilities}{\texttt{xargs} - build and execute commands from stdin}

\begin{itemize}[<+->]
\item
  Example: \texttt{echo\ 1\ 2\ 3\ \textbar{}\ xargs\ echo}
\item
  Example: \texttt{echo\ 1\ 2\ 3\ \textbar{}\ xargs\ -n\ 1\ echo}
\item
  Example: \texttt{echo\ 1\ 2\ 3\ \textbar{}\ xargs\ -P\ 2\ -n\ 1\ echo}
\end{itemize}

\end{frame}
%%%%%%%%%%%%%%%%%%%%%%%%%%%%%

%%%%%%%%%%%%%%%%%%%%%%%%%%%%%%%
\subsection{\tt find}
\begin{frame}{Intermidiate Utilities}{\texttt{find} - search for files in a directory hierarchy}

\begin{itemize}[<+->]
\item
  Example: \texttt{find\ .\ -type\ f\ -name\ \textbackslash{}*.cpp}
\item
  Example:
  \texttt{find\ .\ -type\ d\ -iname\ \textbackslash{}*a\textbackslash{}*\ -exec\ ls\ \{\}}
\item
  Practise:\\
  I want to know how many lines of python code I wrote for a project.\\
  Now I'm in \texttt{/}, and the project is in
  \texttt{\textasciitilde{}/Documents/project/SYSU-Software-2015/}.\\
  What should I do?
\item
  Answer:
  \texttt{find\ \textasciitilde{}/Documents/project/SYSU-Software-2015/\ -type\ f\ -name\ \textbackslash{}*.py\ \textbar{}xargs\ \ wc\ -l}
\end{itemize}

\end{frame}
%%%%%%%%%%%%%%%%%%%%%%%%%%%%%

%%%%%%%%%%%%%%%%%%%%%%%%%%%%%%%
\subsection{\tt grep}
\begin{frame}{Intermidiate Utilities}{\texttt{grep} - find matching lines in files}

\begin{itemize}[<+->]
\item
  Example: \texttt{grep\ iostream\ a.cpp}
\item
  Example: \texttt{grep\ -n\ vector\ a.cpp}
\item
  Example:
  \texttt{find\ .\ -type\ f\ -name\ \textbackslash{}*.py\ \textbar{}\ xargs\ grep\ -ni\ taskhall}
\item
  Example: \texttt{grep\ -\/-include=*.py\ -rni\ taskhall}
\item
  Example:
  \texttt{grep\ -rne\ \textquotesingle{}\textbackslash{}d+\textquotesingle{}}
\end{itemize}

\end{frame}
%%%%%%%%%%%%%%%%%%%%%%%%%%%%%

%%%%%%%%%%%%%%%%%%%%%%%%%%%%%%%
\subsection{\tt sed}
\begin{frame}{Intermidiate Utilities}{\texttt{sed} - line based text processing}

\begin{itemize}[<+->]
\item
  Example:
  \texttt{sed\ -i\ -\/-\ \textquotesingle{}s/127.0.0.1/10.1.10.1/g\textquotesingle{}\ *.conf}
\item
  Example:
  \texttt{sed\ -\/-\ \textquotesingle{}/\^{}\#\#/a\textbackslash{}\%666\textquotesingle{}\ posix\_cli\_tools.md}
\item
  Example:
  \texttt{sed\ -\/-\ \textquotesingle{}/\^{}\#\#/i\textbackslash{}\%\%\%\%\%\%\textquotesingle{}\ posix\_cli\_tools.md}
\end{itemize}

\end{frame}
%%%%%%%%%%%%%%%%%%%%%%%%%%%%%

%%%%%%%%%%%%%%%%%%%%%%%%%%%%%%%
\subsection{\tt diff}
\begin{frame}{Intermidiate Utilities}{\texttt{diff} - find differences between files/folders}
    \begin{itemize}
        \item
            Example:
            \texttt{diff\ a.txt\ b.txt}
        \item
            Example:
            \texttt{diff\ 1.dir\ 2.dir}
    \end{itemize}
\end{frame}
%%%%%%%%%%%%%%%%%%%%%%%%%%%%%

%%%%%%%%%%%%%%%%%%%%%%%%%%%%%%%
\subsection{\tt patch}
\begin{frame}{Intermidiate Utilities}{\texttt{patch} - apply difference file}
    \begin{itemize}
        \item
            Example:
            \texttt{patch orig patch}
    \end{itemize}
\end{frame}
%%%%%%%%%%%%%%%%%%%%%%%%%%%%%

\section{Wonderful Tools}\label{wonderful-tools}

%%%%%%%%%%%%%%%%%%%%%%%%%%%%%%%
\subsection{\tt Foreword}
\begin{frame}{Wonderful Tools}{Some Foreword}

\begin{itemize}
\item
    The tools I am going to introduce to you are not part of \texttt{POSIX}.
\item
    These great tools are not \texttt{*nix}-limited. \texttt{Windows} have them too!
\item
    They all require some time to learn. Fear not, the time you spent on them will pay.
\end{itemize}
\end{frame}
%%%%%%%%%%%%%%%%%%%%%%%%%%%%%

%%%%%%%%%%%%%%%%%%%%%%%%%%%%%%%
\subsection{\tt Editor}
\begin{frame}{Wonderful Tools}{Holy War of Editors: \texttt{vim} or \texttt{emacs}?}

\begin{itemize}[<+->]
\item
  Among numerous choices of editors, \texttt{vim} and \texttt{emacs}
  have distinguished themselves from the others.
\item
  \texttt{vim}: The God of Editor.
\item
  \texttt{emacs}: The Editor of God.
\item
  Demostration.
\end{itemize}

\end{frame}
%%%%%%%%%%%%%%%%%%%%%%%%%%%%%

%%%%%%%%%%%%%%%%%%%%%%%%%%%%%%%
\subsection{\LaTeX{}}
\begin{frame}{Wonderful Tools}{Crafted by the Author of TAOCP: \texttt{TeX}}

\begin{itemize}
\item
  \TeX{} is the most widespread, most advanced and most
  professional typesetting system.
\item
  \LaTeX{} handles content side of document, while \TeX{}
  handles the layout side.\\
  Together, marvelous documents are made.
\end{itemize}

\end{frame}
%%%%%%%%%%%%%%%%%%%%%%%%%%%%%

%%%%%%%%%%%%%%%%%%%%%%%%%%%%%%%
\subsection{\tt pandoc}
\begin{frame}{Wonderful Tools}{Universal Document Converter: \texttt{pandoc}}

\begin{itemize}[<+->]
\item
  Ever fancied conversion between \texttt{docx} and \texttt{html},
  \texttt{markdown} and \texttt{latex}, even \texttt{markdown} and
  \texttt{h5\ slides}?
\item
  \texttt{pandoc} comes to rescue!
\item
  \textbf{These} slides are converted from \texttt{markdown} :-)
\end{itemize}

\end{frame}
%%%%%%%%%%%%%%%%%%%%%%%%%%%%%

%%%%%%%%%%%%%%%%%%%%%%%%%%%%%%%
\subsection{\tt git}
\begin{frame}{Wonderful Tools}{The Most Popular And Easy-to-use VCS: \texttt{git}}

\begin{itemize}[<+->]
\item
  Use \texttt{git}, use \texttt{git}, use \texttt{git}!
\item
  Demonstration.
\end{itemize}

\end{frame}
%%%%%%%%%%%%%%%%%%%%%%%%%%%%%

\section{Conclusion}\label{conslusion}

%%%%%%%%%%%%%%%%%%%%%%%%%%%%%
\begin{frame}{Conclusion}

\begin{itemize}[<+->]
\item
  To learn them, use them to solve real problems (and for fun).
\item
  Never feel ashamed for forgetting the command line options of a command.
  Just use \texttt{--help} or \texttt{man} when you need reference.
\item
  Have fun with CLI!
\end{itemize}

\end{frame}
%%%%%%%%%%%%%%%%%%%%%%%%%%%%%

%%%%%%%%%%%%%%%%%%%%%%%%%%%%%
\begin{frame}{Conclusion}

    \begin{center}
        {\LARGE Thank you!}\\
        ~\\
        Brought to you by \emph{Siyuan Liu}, MSTC, SYSU.\\
        {\tt Nov 22\textsuperscript{nd}, 2015}
    \end{center}

\end{frame}
%%%%%%%%%%%%%%%%%%%%%%%%%%%%%

\end{document}

