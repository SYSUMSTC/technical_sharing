\documentclass{beamer}
\usetheme{Aalborg}

\usepackage{hyperref,graphicx}
\usepackage{verbatim,hyperref}
\usepackage[utf8]{inputenc}
\usepackage{tgheros}
\usepackage[T1]{fontenc}

\begin{document}
\title{An Introduction to Python}
\subtitle{MSTC Technical Sharing}
\author{Richard Tsai}
\institute{
Microsoft Student Technology Club,\\
Sun Yat-sen University}
\date{\today}

\begin{frame}[plain]
\titlepage
\end{frame}

\begin{frame}{Agenda}
 \tableofcontents
\end{frame}

\section{What's Python?}
\subsection{Python the Language}
\begin{frame}{What's Python?}
 \begin{block}{Python is a programming language.}
  \begin{figure}[htb!]
   \includegraphics[scale=0.5]{python-logo-generic}\footnotemark
  \end{figure}
 \end{block}
 \footnotetext{Image from Python Software Foundation}
\end{frame}

\subsection{Easy-to-read}
\begin{frame}{An easy-to-read language}
 \begin{block}{Python is an easy-to-read language.}\pause
  \begin{itemize}
   \item Forced indentation\pause
   \item Less braces\pause
   \item etc.
  \end{itemize}
 \end{block}
\end{frame}


\subsection{Easy-to-write}
\begin{frame}{An easy-to-write language}\pause
 \begin{block}{Python is an easy-to-write language.}\pause
  \begin{itemize}
   \item Dynamic typing\pause
   \item Useful built-in data types\pause
   \item Lots of syntax sugars\pause
   \item Lots of libraries\pause
   \item etc.
  \end{itemize}
 \end{block}
\end{frame}

\begin{frame}[fragile]{A (kidding) example}
Calculate $\displaystyle\sum_{n=1}^{10} n ^ n$ ...
\pause
\begin{block}{Python}
\begin{verbatim}
print sum(n ** n for n in range(1, 11))
\end{verbatim}
\end{block}
\pause
\begin{block}{C++11}
\begin{verbatim}
int64_t seq[10], i = 0;
std::generate_n(seq, 10,
        [&i](){ ++i; return pow(i, i); });
std::cout << std::accumulate(seq, seq + 10,
        static_cast<int64_t>(0)) << std::endl;
\end{verbatim}
\end{block}
\end{frame}

\subsection{Learn Python in 5 minutes}
\begin{frame}[fragile]{Learn Python in 5 minutes}\pause
\begin{columns}
\column{0.5\textwidth}
\begin{block}{Loop}
\begin{verbatim}
for i in range(3):
    print i

for i in [0, 1, 2]:
    print i

while i < 3:
    print i
    i += 1
\end{verbatim}
\end{block}
\pause
\column{0.5\textwidth}
\begin{block}{Condition}
\begin{verbatim}
if a == b:
    print 'a equals to b'
\end{verbatim}
\end{block}
\pause
\begin{block}{Function}
\begin{verbatim}
def f(n):
    return n + 1

print f(1)
\end{verbatim}
\end{block}

\end{columns}
\end{frame}

\begin{frame}[fragile]{Learn Python in 5 minutes}
\begin{columns}
\column{0.5\textwidth}
\begin{block}{Data types}
\begin{verbatim}
l = ['one', 2, 3]
print l[0] # one
print l[1:3] # [2, 3]

t = ('one', 2, 3)
print l[1] # one

d = {'one': 1, 'two': 2}
print d['two'] # 2

s = {'one', 2, 3}
if 3 in s:
    print '3 is in s'
\end{verbatim}
\end{block}
\pause
\column{0.5\textwidth}
\begin{block}{Packages}
\begin{verbatim}
import math
math.sin(math.pi / 2)
\end{verbatim}
\end{block}
\pause
\begin{block}{Run a Python program}
\begin{itemize}
 \item Interactive interpreter\newline
 (IPython recommended)
 \item Run as a script
\end{itemize}
\end{block}
\end{columns}
\end{frame}

\section{What can Python do?}
\subsection{Some applications}
\begin{frame}{What can Python do?}\pause
\begin{block}{Some applications}
 \begin{itemize}
  \item As a shell script language\pause
  \item Web apps (WSGI, Flask, Django, Tornado...)\pause
  \item Scientific calculations (Numpy, Scipy...)\pause
  \item Parallel computing (multiprocessing, pp, mpi4py...)\pause
  \item Many other things you can and cannot imagine...
 \end{itemize}
\end{block}
\end{frame}

\section{When (not) to use Python?}
\subsection{Drawbacks of Python}
\begin{frame}{When (not) to use Python?}\pause
 \begin{block}{Drawbacks of Python}
  \begin{itemize}
   \item Performance\pause
   \item Fxxking GIL\pause
   \item Hard to debug\pause
   \item Suck documents (to some extent)\pause
   \item Odd support in functional programming (IMO)\pause
  \end{itemize}
 \end{block}
 
 \begin{block}{When (not) to use Python?}
 \end{block}
\end{frame}

\section{Accelerate Python}
\begin{frame}{Accelerate Python}\pause
 \begin{block}{Pareto principle}\pause
 \end{block}
 \begin{block}{Accelerate Python with other languages}
  \begin{itemize}
   \item Third-party libraries
   \item Python C API
   \item SWIG, SIP...
   \item ctypes
   \item Cython
  \end{itemize}
 \end{block}
\end{frame}

\section{Resources}
\begin{frame}{Resources}
 \begin{itemize}
  \item Python's official website: \url{http://www.python.org}
  \item Beginner's Guide to Python: \url{https://wiki.python.org/moin/BeginnersGuide}
 \end{itemize}
\end{frame}

\section*{Other things}
\begin{frame}{Q\&A}
\begin{center}
 \Huge Q\&A
\end{center}
\end{frame}

\begin{frame}{Thanks}
\begin{center}
 \Huge Thanks!
\end{center}
\end{frame}

\end{document}
